\subsubsection{Masked Projection for Initialization}
\label{method22}


While our model handles procedure planning tasks effectively, the denoising sampling process in diffusion models does not always guarantee that the generated actions fall within the desired range. To mitigate this issue, we introduce a mask projection that constrains the action space during the denoising process. This approach is inspired by the masked latent modeling scheme proposed by~\citet{gao2023masked}.


Since each task has a specific action scope, we activate only the actions associated with the current task class label $c$ and deactivate the others. When constructing the input matrix $\hat{x}_N$ for the denoising process, we add the initial Gaussian noise solely to the active action positions, while setting the non-active action positions to zero. This process can be expressed as follows:
\begin{equation}
\hat{a}_{t,d} = 
\begin{cases} 
      \epsilon, & \text{if } d \in Task(c) \\
      0, & \text{if } d \notin Task(c)
\end{cases},
\end{equation}
where $d$ denotes the action ID spanning the action dimension $A$, and $\epsilon \sim \mathcal{N}(0,1)$. The function $Task(c)$ represents the set of actions corresponding to task $c$. By restricting the initial noise in the input matrix $\hat{x}_N$ to the relevant action scope, the model ensures that the procedure planning is confined to the active actions.


\subsection{MTID: Masked Temporal Interpolation Diffusion}
\label{method2} 
Our approach consists of three main stages: predicting the task class, obtaining latent temporal interpolation supervision to guide the logical sequence, and modeling the distribution of action sequences. Unlike \citet{wang2023pdpp}, the first stage involves solving a standard classification problem using a multi-head self-attention transformer to extract features from observations and transform them into class labels. This process includes linear transformations, ReLU activations, and dropout layers to enhance the model's expressiveness and generalization capabilities.

The second stage leverages \(V_{start}\) and \(V_{goal}\) as inputs for the latent space temporal interpolation module to obtain the latent frames needed to solve \(p(\upsilon_{1:n} \mid V_{start}, V_{goal})\). These frames are then used for cross-attention within the residual blocks of the U-Net architecture. During the cross-attention calculation, the latent information obtained from the interpolation module serves as conditioning information to adjust the intermediate matrices.

The third stage focuses on modeling \(p(a_{1:T} \mid \upsilon_{1:n}, c, V_{start}, V_{goal})\) to effectively handle the procedure planning task.

\subsubsection{Diffusion Models and DDIM (Denoising Diffusion Implicit Models)}
Diffusion models are generative models that learn to represent data distributions \(p(x_0)\) by progressively adding noise to the data in a series of steps, producing a sequence of latent variables \(\{x_N, \ldots, x_0\}\). The standard diffusion process is a Markovian process~\citep{ho2020denoising,nichol2021improved}, where noise is added at each step of the forward process and removed during the reverse process, ultimately generating new samples by reversing the added noise.

Denoising Diffusion Implicit Models (DDIM)~\citep{song2020denoising} are a variation of the traditional diffusion models that introduce a non-Markovian process to improve the efficiency of sampling. DDIM modifies the forward and reverse processes to be deterministic, maintaining a direct mapping between \(x_n\) and \(x_0\) without adding stochastic noise at each step. This approach enables faster sampling with fewer diffusion steps, while still producing high-quality samples.

In DDIM, the forward process is re-parameterized as:
\begin{equation}
\begin{aligned}
x_n = \sqrt{\bar{\alpha}_n} \, x_0 + \sqrt{1 - \bar{\alpha}_n} \, \epsilon ,
\end{aligned}
\end{equation}
where \(\bar{\alpha}_n = \prod_{s=1}^n \left(1 - \beta_s\right)\), and \(\epsilon \sim \mathcal{N}(0, I)\). This equation is used for both the forward and reverse processes, which ensures deterministic transitions between the steps.

The reverse process in DDIM is defined by:
\begin{equation}
\begin{aligned}
x_{n-1} = \sqrt{\bar{\alpha}_{n-1}} \, x_0 + \sqrt{1 - \bar{\alpha}_{n-1}} \, \epsilon_\theta\left(x_n, n\right) ,
\end{aligned}
\end{equation}
where \(\epsilon_\theta\left(x_n, n\right)\) is the predicted noise at step \(n\), parameterized by a neural network. The deterministic nature of DDIM allows it to sample more efficiently compared to traditional diffusion models, reducing the number of necessary diffusion steps while maintaining or even improving the sample quality.

In this work, we utilize DDIM due to its ability to strike a balance between sample generation speed and flexibility, enabling us to generate samples faster with fewer diffusion steps, without sacrificing performance compared to the standard diffusion model.


\subsubsection{Masked Projection for Initial and Iterative Steps}
In a typical diffusion model, the input consists of the data distribution the model is designed to fit, without requiring any additional guidance. However, for our procedure planning task, the distribution we aim to model consists of intermediate action sequences \([a_1, a_2, \dots, a_T]\), which are influenced by initial observations and the task class derived from an earlier learning stage. This requires a method to incorporate these guiding conditions into the diffusion process.

We construct the input matrix \( \hat{x}_i \) by concatenating three components: (1) the visual observations of the start and goal states \((V_s, V_g)\), (2) the predicted task class \( c \), and (3) a sequence of candidate actions \((a_{1:T})\). \citet{wang2023pdpp} incorporate these guiding conditions as supplementary inputs by concatenating them directly to the action features. Specifically, they concatenate the starting observation \( o_s \) with the first action \( a_1 \), and the goal observation \( o_g \) with the last action \( a_T \). This method assumes that the start and goal observations are most relevant to the first and last actions, respectively, providing prior knowledge to the model. Intermediate observations are not used as supervision signals, so their values are set to zero except for the start and goal observations. Thus, the input matrix can be expressed as:
\begin{equation}
    \hat{x}_i = \begin{bmatrix}  
        c & c & \cdots & c & c \\  
        \hat{a}_1 & \hat{a}_2 & \cdots & \hat{a}_{T-1} & \hat{a}_T \\  
        V_s & 0 & \cdots & 0 & V_g
    \end{bmatrix}.
\end{equation}
Inspired by the masked latent modeling scheme and asymmetric masking diffusion transformer introduced by \citet{gao2023masked}, which predict masked tokens from unmasked ones during the diffusion process, we design a masking mechanism to enhance contextual relation learning and limit the scope of action generation.

During initial noise addition, Gaussian noise is introduced exclusively to the unmasked (active) actions. This strategy limits the search space for optimal actions to the task-defined subset, reducing the learning load during loss minimization. As the action space grows, this approach becomes more advantageous, leading to faster convergence and higher accuracy in the denoising phase.

In the iterative steps, for \( \varepsilon_i \in \mathbb{R}^{B \times H \times A} \), a binary mask is appended to the candidate action sequence. This mask is determined by the predicted task class \( c \), with `1' indicating active actions in the predicted task class, and `0' for other actions along the \( A \) dimension. The masked projection is then defined as:
\begin{equation}
    MP(\varepsilon_i, c) = \hat{a}_i,
\end{equation}
where \( B \) denotes batch size, \( H \) denotes horizon steps, \( A \) denotes action dimensions, and \( \varepsilon \) is sampled from random noise. By sampling from random noise, concatenating the guiding conditions, and applying the masked projection, the input matrix \( \hat{x}_i \) is constructed for the diffusion process.



\subsubsection{Latent Space Temporal Interpolation}
In this subsection, we introduce how our method uses temporal interpolation to guide the iteration of U-Net. The key to addressing the challenge of maintaining the logical continuity and temporal coherence within horizon steps is establishing connections between the actions and observations during the iteration process. 

To solve this problem, we propose the \textit{Temporal Interpolation Predictor}, which encodes visual observations into a latent space that captures temporal and logical relationships, enabling more accurate action predictions. Specifically, our Temporal Interpolation Predictor uses an encoder $E$ with convolutional layers to transform visual observations into a latent temporal space:
\begin{equation}
L_s, L_g = E(V_s, V_g),
\end{equation}
where $L_s$ and $L_g$ represent the latent features of the visual observations $V_s$ and $V_g$.

In the latent space, we employ a transformer-based structure to predict several intermediate latent features. First, the latent features $L_s$ and $L_g$ are linearly interpolated to form a sequence \{ $I_1, I_2, \dots, I_n$ \}, where $n$ is an adjustable hyper-parameter. 

The alpha generation is defined by:
\begin{equation}
\alpha = \sigma(W \cdot \mathbf{1} + b),
\end{equation}
where $\sigma$ is the sigmoid function, $W \in \mathbb{R}^{b \times d}$ is the weight matrix, $\mathbf{1} \in \mathbb{R}^d$ is a vector of ones, $b \in \mathbb{R}^b$ is the bias vector, and $\alpha \in \mathbb{R}^{b \times n}$ represents the alpha values, with $b$ as the batch size and $n$ as the number of blocks.

The interpolation for each frame is computed as:
\begin{equation}
I_i = (1 - \alpha_i) \cdot x_1 + \alpha_i \cdot x_2,
\end{equation}
where $I_i \in \mathbb{R}^{b \times d}$ is the $i$-th interpolated frame, $x_1, x_2 \in \mathbb{R}^{b \times d}$ are the input frames, and $\alpha_i \in \mathbb{R}^{b \times 1}$ is the $i$-th column of $\alpha$. The alpha values control the relative weighting of the two input frames in each interpolated frame.

This sequence is then passed through a series of transformer blocks $B$ to predict the transition latent features:
\begin{equation}
F_1, F_2, \dots, F_n = B(I_1, I_2, \dots, I_n).
\end{equation}
Motivated by \citet{khachatryan2023text2video}, we incorporate cross-attention layers into the residual blocks of a U-Net. We use the transition latent features as keys and values, and the original $\hat{X}_i$ inside the model as queries. The cross-attention is computed as:
\begin{equation}
\hat{X}_i = \text{CrossAttention}(\hat{X}_i, F_i, F_i).
\end{equation}


% \subsubsection{Semantic Motion Predictor for Video Generation}
% We argue that this limitation arises from their reliance on temporal modules, which predict intermediate frames based solely on temporal dynamics, often disregarding spatial information.

% Similar to previous video generation methods, we optimize the model by computing the MSE loss between the predicted video frames $O = (O_1, O_2, \dots, O_L)$ and the ground truth frames $G = (G_1, G_2, \dots, G_L)$:

% \begin{equation}
% \mathrm{Loss} = \mathrm{MSE}(G, O).
% \end{equation}

% By encoding images into a semantic space that captures spatial relationships, our Semantic Motion Predictor effectively models motion information, resulting in smooth transition videos even with large motion. The improvements can be observed in Fig. 1 and Fig. 5.


% \subsubsection{Training-Free Consistent Images Generation}

% We insert Consistent Self-Attention in place of the original self-attention layer in the U-Net architecture of existing image generation models. To maintain the training-free and pluggable nature, we reuse the original self-attention weights. 

% Formally, given a batch of image features $I \in \mathbb{R}^{B \times N \times C}$, where $B$, $N$, and $C$ represent the batch size, number of tokens in each image, and channel number, respectively, we define a function $\mathrm{Attention}(X_k, X_q, X_v)$ to compute self-attention. $X_k$, $X_q$, and $X_v$ represent the key, query, and value matrices used in the attention calculation. The original self-attention is performed independently on each image feature $I_i \in I$. The feature $I_i$ is projected to $Q_i$, $K_i$, $V_i$, and sent into the attention function, yielding:

% \begin{equation}
% O_i = \operatorname{Attention}\left( Q_i, K_i, V_i \right).
% \end{equation}

% To enable interactions among images within a batch for subject consistency, our Consistent Self-Attention samples some tokens $S_i$ from other image features in the batch:

% \begin{equation}
% S_i = \operatorname{RandSample}\left( I_1, I_2, \dots, I_{i-1}, I_{i+1}, \dots, I_B \right),
% \end{equation}

% where $\operatorname{RandSample}$ is the random sampling function. After sampling, we pair the sampled tokens $S_i$ with the image feature $I_i$ to form a new set of tokens $P_i$. We perform linear projections on $P_i$ to generate new key $K_{P_i}$ and value $V_{P_i}$ for Consistent Self-Attention. The original query $Q_i$ remains unchanged. Finally, the self-attention is computed as follows:

% \begin{equation}
% O_i = \operatorname{Attention}\left( Q_i, K_{P_i}, V_{P_i} \right).
% \end{equation}

% By performing self-attention across the batch, our method facilitates interactions among features of different images. This promotes the convergence of characters, faces, and attires during the generation process. Despite its simplicity and training-free nature, our Consistent Self-Attention efficiently generates subject-consistent images, as we will demonstrate in our experiments. These images help narrate a complex story, as shown in Fig. 2. For clarity, we also provide the pseudo code in Algorithm 1.